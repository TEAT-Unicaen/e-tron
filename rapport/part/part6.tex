\section{Analyse des Fonctionnalités et Résultats de la Page React}

\subsection{Fonctionnalités de la Page React}
La page React développée permet d'analyser les résultats des parties de Tron jouées par les algorithmes. Elle repose sur le chargement d'un fichier JSON généré après chaque partie, et offre plusieurs fonctionnalités pour visualiser et interpréter les performances des IA. \\
\textbf{1. Chargement et Analyse des Données}
\begin{itemize}
    \item La page accepte un fichier JSON contenant les données de la partie.
    \item Elle extrait et affiche les statistiques des joueurs sous forme de graphiques et de timeline.
\end{itemize}
\textbf{2. Visualisation des Tendances de Déplacement}
\begin{itemize}
    \item Un graphique radar montre les moyennes des mouvements effectués par chaque joueur.
    \item Les tendances de déplacement sont représentées pour comparer le comportement des IA.
    \item Il est possible d'afficher ou masquer les statistiques d'un joueur en activant/désactivant son affichage.
\end{itemize}
\textbf{3. Timeline des Actions}
\begin{itemize}
    \item Une seconde page permet de voir les actions des joueurs tour par tour.
    \item Un curseur permet de sélectionner une plage de tours à afficher.
    \item Les actions peuvent être filtrées par les joueurs.
\end{itemize}

\subsection{Analyse des Résultats}
\subsubsection{Identification d'un Seuil d'Exploration}
\subsubsection{Mesure des Performances en Fonction de la Profondeur}
\subsubsection{Équilibre entre Coût et Gain Stratégique}