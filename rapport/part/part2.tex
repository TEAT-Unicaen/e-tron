\section{Objectif du projet}
\subsection{La Problématique}
Quelle influence la profondeur de recherche a-t-elle sur le jeu en fonction de la taille des équipes et de la taille de la grille ?

\subsection{Nos intérrogations}
Une question centrale est de savoir s’il existe un seuil de profondeur au-delà duquel la recherche devient inefficace. À partir de quel point l’exploration supplémentaire n’apporte-t-elle plus d’amélioration significative sur les décisions de nos IA ? Et ce seuil dépend-il de la taille de la grille et/ou du nombre de joueurs ?
Si un tel seuil existe, comment le déterminer ? Peut-on l’identifier en mesurant l’évolution des performances en fonction de la profondeur ? Est-il fixe, ou bien varie-t-il selon la situation en cours  ?
Enfin, à quel moment la recherche devient-elle trop coûteuse par rapport au gain stratégique ? Augmenter la profondeur améliore-t-il toujours la qualité du jeu, ou atteint-on un point où le coût dépasse l’intérêt tactique ? Existe-t-il un compromis optimal entre précision et rapidité pour garantir de bonnes décisions sans alourdir excessivement les calculs ?

\subsection{Nos ambitions}
Dès le départ, nous avons voulu nous lancer un défi en choisissant \textbf{C++} pour ce projet. Plutôt que d'opter pour une solution plus simple, nous avons voulu exploiter un langage plus exigeant afin de mieux comprendre les rouages d’un moteur de jeu et de gagner en maîtrise sur la gestion des performances. \\
L'idée d'utiliser \textbf{DirectX} est d’ailleurs née d’une blague entre nous. Ce qui devait être une simple plaisanterie s’est transformé en un véritable challenge que nous avons décidé de relever. Finalement, ce choix nous a permis d'explorer en profondeur les aspects techniques du rendu 3D, la gestion de la mémoire et l'optimisation des performances. \\
En plus du moteur de jeu, nous avons voulu aller plus loin en développant un \textbf{outil d’analyse} pour mieux comprendre les performances de nos algorithmes et leurs prises de décision. Cela nous a amenés à concevoir une interface permettant d'afficher visuellement les tendances de déplacement et d'étudier l’impact des différentes stratégies d’IA. \\
