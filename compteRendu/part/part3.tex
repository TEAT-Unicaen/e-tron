\section{Quels algorithmes pour répondre ?}
Leur utilisation, leur fonctionnement et leur lien avec le projet
Dans notre projet de jeu de Tron, l'intelligence artificielle joue un rôle central. Nous avons exploré plusieurs approches pour permettre aux agents de prendre des décisions stratégiques. Deux familles d'algorithmes se sont avérées particulièrement pertinentes : \textbf{Minimax} (avec ses variantes \textbf{MAXN} et \textbf{Paranoid}) et une méthode inspirée de \textbf{SOS (Socially Oriented Search)} pour la gestion des équipes. Ces méthodes étaient explicitement demandées dans les attendus du projet.

\subsection{Minimax}
Le Minimax est un algorithme décisionnel classiquement utilisé dans les jeux à information parfaite comme les échecs ou le go. Il repose sur une évaluation récursive des coups possibles, cherchant à minimiser les pertes tout en maximisant les gains. Cependant, dans un jeu comme Tron, qui peut inclure plusieurs joueurs et des interactions complexes, nous avons exploré deux variantes plus adaptées : \textbf{MAXN} et \textbf{Paranoid}.
Un seul de ces deux algorithmes était requis, mais nous avons pris la décision d'implémenter les deux dans notre projet car le temps nous le permettait.

\subsubsection{MAXN}
Le Minimax standard est conçu pour des jeux à deux joueurs, alternant entre un joueur qui maximise et un joueur qui minimise. Or, dans un jeu multi-joueur comme Tron, il faut un algorithme capable d'évaluer plusieurs adversaires simultanément.
Le MAXN est une extension naturelle du Minimax aux jeux à plus de deux joueurs. Son fonctionnement repose sur une évaluation de chaque action non pas sous un simple critère de maximisation/minimisation binaire, mais en attribuant une valeur à chaque joueur.
\vspace{0.3cm}

\textbf{Utilisation dans notre projet :}
\begin{itemize}
    \item Chaque joueur dans la simulation a une fonction d'évaluation qui estime son score potentiel à partir du nombre de positions “sécuritaires” autour de lui (nombre de cases accessibles).
    \item L'algorithme explore l'arborescence des coups possibles, attribuant un vecteur de scores (un par joueur) à chaque état du jeu.
    \item Il prend ensuite la décision qui maximise son propre score sans considérer spécialement un adversaire comme principal opposant.
\end{itemize}
Cette approche permet une prise de décision plus naturelle en situation multi-joueur, mais elle suppose que chaque joueur agit de manière strictement rationnelle, ce qui n'est pas toujours le cas dans une partie réelle.

\subsubsection{Paranoid}
Contrairement à MAXN, qui considère chaque joueur comme une entité indépendante maximisant son propre score, \textbf{Paranoid} adapte Minimax à un cadre multi-joueur en supposant que tous les adversaires sont alignés contre le joueur actif.
Dans cette approche :
\begin{itemize}
    \item Le joueur actif maximise son gain.
    \item Tous les autres joueurs sont considérés comme un seul agent combiné, jouant de manière coopérative pour réduire son score.
\end{itemize}
\textbf{Pourquoi utiliser Paranoid ?}
\begin{itemize}
    \item Dans Tron, un joueur peut se retrouver encerclé par plusieurs adversaires. Dans ce cas, considérer les autres comme un groupe homogène hostile permet d'anticiper les pires scénarios et d'opter pour des stratégies plus prudentes
    \item Cette approche peut être plus efficace dans un cadre où la survie est prioritaire sur le score brut.
\end{itemize}

\subsubsection{Comparaison entre MAXN et Paranoid}
\begin{itemize}
    \item \textbf{MAXN} offre une modélisation plus fidèle d'un jeu multi-joueur libre, mais peut sous-estimer les alliances tacites.
    \item \textbf{Paranoid} prépare mieux à des situations hostiles, mais peut donner des résultats trop défensifs.
\end{itemize}
Dans notre projet, nous avons implémenté les deux méthodes car nous avons estimé qu'il valait mieux disposer d'un large éventail de résultats statistiques. Nous analyserons ultérieurement les performances de ces approches et verrons si nos hypothèses sur leurs avantages respectifs sont confirmées par les résultats expérimentaux.

\subsection{Spécifier des équipes en s'inspirant de SOS}
En plus des algorithmes de prise de décision individuelle, nous avons cherché un moyen d'organiser les joueurs en équipes tout en conservant un système de prise de décision efficace. Pour cela, nous nous sommes inspirés du concept \textbf{SOS (Socially Oriented Search)}, couramment utilisé dans l'optimisation et la théorie des jeux. Cette approche faisait également partie des attendus du projet.
\textbf{Pourquoi SOS ?}
\begin{itemize}
    \item SOS permet de formuler des contraintes de déséquilibre entre différents agents, créant des comportements agressifs ou préventifs.
    \item Simple d’utilisation, il permet d’appliquer un poids à chaque possibilité, en fonction de sa dangerosité, à savoir, la proximité d’un ennemi.
\end{itemize}

\newpage

\textbf{Application dans notre projet :}
\begin{itemize}
    \item Une matrice d’affinités peut être générée aléatoirement ou spécifiée, afin de générer des équipes
    \item Lors de chaque tour, l’algorithme MAXN, modifié pour intégrer SOS va lister normalement les chemins possibles par score puis ajoute un multiplicateur, une pondération, à chaque chemin, en fonction du risque représenté.
    \item Cela permet de préserver la survie d’un joueur face à un adversaire agressif, prioritairement à un chemin à possibilité élevée.
\end{itemize}
Cette approche, bien que différente des résolutions classiques des jeux de stratégie, nous a permis d'obtenir des parties plus dynamiques et plus intéressantes à observer. Nous évaluerons ultérieurement l'impact réel de ce modèle sur la compétitivité et l'équilibre des parties.

\subsection{Conclusion provisoire}
L'IA dans notre jeu de Tron repose donc sur des algorithmes avancés de prise de décision et d'organisation des joueurs. \textbf{MAXN} et \textbf{Paranoid} permettent d'aborder différemment les défis du jeu multi-joueur, tandis que l'utilisation d'une méthode inspirée de \textbf{SOS} nous a permis de structurer les équipes de façon efficace.
Ces approches faisaient partie des exigences du projet et ont été implémentées pour évaluer leurs performances respectives. Nous verrons par la suite si nos estimations initiales concernant leur efficacité et leur pertinence se confirment avec les résultats obtenus en simulation.
