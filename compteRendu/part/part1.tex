\section{Jeu de Tron multi-joueur et coalitions}
\subsection{Description du projet}
Notre projet vise à implémenter une version multijoueur du jeu de Tron en C++ avec une visualisation en 3D sous DirectX11. L'IA repose sur des algorithmes de recherche. L'objectif est d'analyser l'impact de la profondeur de recherche sur les performances des équipes en fonction de la taille de la grille et de la composition des groupes.

\subsection{Répartition des tâches}
Pour réaliser ce projet nous nous sommes réparti les tâches. Amand a été chargé de s'occuper de la transcription des algorithmes pseudo-code dans le jeu de Tron, Théo a réalisé la structure du jeu et s'est occupé de la question scientifique en analysant les données obtenues par les algorithmes et en affichant les résultats dans une page en react. Tandis que Jonathan et Tom on pu s'occuper d'une partie visuel avec une fenêtre sous Windows32 et un moteur de jeu en DirectX11.

\subsection{Plan du rapport}
\tableofcontents

\subsection{Répartition des tâches}
\begin{itemize}
    \item Amand a été chargé de s'occuper de la transcription des algorithmes pseudo-code dans le jeu de Tron
    \item Théo a réalisé la structure du jeu et s'est occupé de la question scientifique en analysant les données obtenues par les algorithmes et en affichant les résultats dans une page en react
    \item Jonathan et Tom on pu s'occuper d'une partie visuel avec une fenêtre sous Windows32 et un moteur de jeu en DirectX11
\end{itemize}