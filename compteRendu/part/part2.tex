\section{Objectif du projet}
\subsection{La Problématique}
Quelle influence la profondeur de recherche a-t-elle sur le jeu en fonction de la taille des équipes et de la taille de la grille ?

\subsection{Nos intérrogations}
Une question centrale est de savoir s’il existe un seuil de profondeur au-delà duquel la recherche devient inefficace. À partir de quel point l’exploration supplémentaire n’apporte-t-elle plus d’amélioration significative sur les décisions de nos IA ? Et ce seuil dépend-il de la taille de la grille et/ou du nombre de joueurs ?
Si un tel seuil existe, comment le déterminer ? Peut-on l’identifier en mesurant l’évolution des performances en fonction de la profondeur ? Est-il fixe, ou bien varie-t-il selon la situation en cours  ?
Enfin, à quel moment la recherche devient-elle trop coûteuse par rapport au gain stratégique ? Augmenter la profondeur améliore-t-il toujours la qualité du jeu, ou atteint-on un point où le coût dépasse l’intérêt tactique ? Existe-t-il un compromis optimal entre précision et rapidité pour garantir de bonnes décisions sans alourdir excessivement les calculs ?